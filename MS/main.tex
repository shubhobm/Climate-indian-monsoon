\documentclass[fleqn,11pt]{article}
\usepackage{mycommands,amssymb,amsmath,amsthm,color,outlines,cite,caption,url,subfigure}
\usepackage[pdftex]{graphicx}
\usepackage[round]{natbib}
\usepackage[pdftex]{epsfig}
\addtolength{\evensidemargin}{-.5in}
\addtolength{\oddsidemargin}{-.5in}
\addtolength{\textwidth}{0.9in}
\addtolength{\textheight}{0.9in}
\addtolength{\topmargin}{-.4in}

\usepackage{hyperref} % for linking references 
\usepackage{setspace}
\doublespacing

%\DeclareMathOperator*{\vec}{vec}
\DeclareMathOperator*{\diag}{diag}
\DeclareMathOperator*{\Tr}{Tr}

\begin{document}

\newtheorem{Theorem}{Theorem}[section]
\newtheorem{Lemma}[Theorem]{Lemma}
\newtheorem{Corollary}[Theorem]{Corollary}
\newtheorem{Proposition}[Theorem]{Proposition}
\theoremstyle{definition} \newtheorem{Definition}[Theorem]{Definition}

\title{Depth model selection outputs using LMM and wild bootstrap}
\maketitle

\section{Variables}
Latitude, Longitude, Elevation, Tmax, Tmin, DelTT, DMI, Nino3.4;\\
u-wind at 200, 600 and 850;\\
v-wind at 200, 600, 850;\\
10 indices of Madden-Julian Oscillations: 20E, 70E, 80E, 100E, 120E, 140E, 160E, 120W, 40W, 10W;\\
Teleconnections: North Atlantic Oscillation (NAO), East Atlantic (EA), West Pacific (WP), East Pacific-North Pacific (EPNP), Pacific/North American (PNA), East Atlantic/Western Russia (EAWR), Scandinavia (SCA), Tropical/Northern Hemisphere (TNH), Polar/Eurasia (POL);\\
Solar Flux;\\
Land-Ocean Temperature Anomaly

\textbf{Total 35 variables}. Medians of precipitation and all these variables are taken across stations and year, and log of precipitation is modeled as LMM with all variables as fixed effects and yearwise random effect.

\section{Bootstrap scheme}
We are assuming the model $Y = X\bfbeta + Z\bfgamma + \bfepsilon$. We do bootstrap on the random effect and residuals separately, i.e. bootstrap samples are calculated as:

$$ Y_b = X\bfbeta + Z\bfgamma_{b_1} + \bfepsilon_{b_2} = X\bfbeta + Z U_{b_1}\bfgamma + U_{b_2}\bfepsilon $$

Where diagonals of $U_{b_1}$ are mean 0 variance $\tau_k^2$ and $U_{b_2}$ are mean 0 variance $\tau_N^2$: $k$ and $N$ being number of classes and number of samples, respectively. Number of bootstrap samples is 1000.

\section{Model selection}
The model selection procedure is as given below:
\begin{enumerate}
\item Compute depth-based model criterion $C_n$ for the full model;
\item Drop a predictor, compute $C_n$ for the reduced model: repeat for all predictors;
\item Collect the predictors dropping which causes a \textit{decrease} in $C_n$. Build final model on these predictors.
\end{enumerate}	

Criterion values for all drop-1 models as well as full model (\texttt{$<$none$>$}) are given in the table below and sorted in increasing order. Also (experimental) since $C_n$ is an expected value and we are comparing whether one mean is less than another, I thought of calculating $p$-values for each drop-1-vs.-full model comparison, by $t$-test with less than type alternative. They are given in third column of the table.

\paragraph{Scheme I: $\tau_n^2 = n^{0.1}$}

\singlespacing
\begin{footnotesize}
\begin{verbatim}
             DroppedVar        Cn        pValue
1                - TMAX 0.1311765  0.000000e+00
2           - ELEVATION 0.1567890  0.000000e+00
3         - TempAnomaly 0.1744005 5.720136e-254
4  - del_TT_Deg_Celsius 0.2107309  2.157130e-84
5              - Nino34 0.2407754  2.446939e-08
6          - v_wind_850 0.2409109  2.168531e-08
7                 - POL 0.2434999  9.084717e-06
8           - SolarFlux 0.2437457  2.064007e-05
9                - EPNP 0.2440997  2.423362e-05
10              - X120W 0.2443009  7.781675e-06
11          - LONGITUDE 0.2453031  2.852942e-04
12         - u_wind_850 0.2461120  9.066987e-04
13                - TNH 0.2471282  3.579881e-03
14                 - EA 0.2477579  8.112529e-03
15         - u_wind_600 0.2479662  1.029812e-02
16           - LATITUDE 0.2503148  1.000400e-01
17         - u_wind_200 0.2506321  1.297290e-01
18                - NAO 0.2519853  2.910135e-01
19                - DMI 0.2520626  2.973663e-01
20               - X20E 0.2523445  3.369725e-01
21               <none> 0.2532943  1.000000e+00
22         - v_wind_200 0.2537721  5.800021e-01
23               - EAWR 0.2545490  7.012957e-01
24         - v_wind_600 0.2554020  8.148190e-01
25                 - WP 0.2557333  8.499471e-01
26               - TMIN 0.2558768  8.606282e-01
27              - X160E 0.2559359  8.735409e-01
28                - PNA 0.2569771  9.399411e-01
29              - X140E 0.2580453  9.789425e-01
30              - X120E 0.2615192  9.997137e-01
31                - SCA 0.2616763  9.997404e-01
32               - X40W 0.2623092  9.999299e-01
33               - X70E 0.2630892  9.999868e-01
34              - X100E 0.2648561  9.999996e-01
35               - X10W 0.2655411  9.999997e-01
36               - X80E 0.2732275  1.000000e+00
\end{verbatim}
\end{footnotesize}

\doublespacing
\paragraph{Scheme II: $\tau_n^2 = n^{0.2}$}

\singlespacing
\begin{footnotesize}
\begin{verbatim}
             DroppedVar        Cn        pValue
1                - TMAX 0.1594159 1.924099e-318
2           - ELEVATION 0.1831007 5.087412e-209
3         - TempAnomaly 0.1950830 1.610033e-149
4  - del_TT_Deg_Celsius 0.2272400  2.230674e-32
5          - v_wind_850 0.2445276  5.883643e-05
6           - LONGITUDE 0.2461506  7.222295e-04
7              - Nino34 0.2463974  1.399145e-03
8           - SolarFlux 0.2474990  6.114477e-03
9          - u_wind_850 0.2482293  1.402247e-02
10                - POL 0.2488754  2.575858e-02
11         - u_wind_600 0.2494258  4.762298e-02
12               - EPNP 0.2495537  4.843258e-02
13               - EAWR 0.2512024  1.698848e-01
14           - LATITUDE 0.2514350  2.025272e-01
15                 - EA 0.2515075  2.070797e-01
16               - TMIN 0.2525158  3.550617e-01
17         - v_wind_200 0.2525735  3.644007e-01
18                - NAO 0.2526962  3.821308e-01
19         - v_wind_600 0.2527427  3.920411e-01
20              - X120W 0.2528025  3.941054e-01
21                - TNH 0.2530542  4.405466e-01
22         - u_wind_200 0.2530875  4.486618e-01
23               <none> 0.2533959  1.000000e+00
24                - PNA 0.2537365  5.583152e-01
25                 - WP 0.2558643  8.517429e-01
26               - X20E 0.2562980  8.970100e-01
27                - DMI 0.2564950  9.070440e-01
28              - X160E 0.2595774  9.955072e-01
29                - SCA 0.2597392  9.960893e-01
30              - X140E 0.2601329  9.977884e-01
31              - X120E 0.2605941  9.986716e-01
32               - X70E 0.2618301  9.997547e-01
33               - X40W 0.2624374  9.999073e-01
34              - X100E 0.2633385  9.999840e-01
35               - X10W 0.2641683  9.999959e-01
36               - X80E 0.2733044  1.000000e+00
\end{verbatim}
\end{footnotesize}

\section{Discussion}
\begin{itemize}
\item Top 5-6 variables are as expected;
\item EPNP teleconnection and 120W MJO (X120W) are both selected in the model. Both deal with the same longitude region... are they related?
\item Interesting variables: Solar Flux and Polar/Eurasia teleconnection (POL): an indicator of Eurasian snow cover;
\item Temperature Anomaly has a huge effect. If we do not include this variable, some MJOs are selected, particularly 80E and 40W get selected consistently (Indian ocean and Atlantic oscillations?) across bootstrap schemes. But including temp anomaly makes MJOs almost expendable;
\item A thing about $p$-values: don't really know how much they are useful here, but when I tried the method for linear model selection on a simulated dataset with 25 predictors from Charlie's webpage (\url{http://www.stat.umn.edu/geyer/5102/examp/select.html}) all variables with true non-zero coeffs had p-values $<0.05$. Here is its output ($\tau_n^2 = n^{0.2}$):

\singlespacing
\begin{footnotesize}
\begin{verbatim}
   DroppedVar        Cn       pValue
1        - x2 0.2051023 1.698015e-42
2        - x3 0.2164894 1.192099e-10
3        - x4 0.2169777 1.211069e-09
4        - x1 0.2209519 1.723179e-04
5        - x5 0.2215058 6.844795e-04
6       - x20 0.2224559 4.012763e-03
7       - x21 0.2252643 1.730813e-01
8        - x9 0.2257273 2.498471e-01
9      <none> 0.2268667 1.000000e+00
10      - x22 0.2279793 7.439175e-01
11      - x17 0.2285479 8.406974e-01
12      - x10 0.2285719 8.426074e-01
13       - x6 0.2290502 8.971758e-01
14      - x13 0.2292558 9.147554e-01
15      - x19 0.2293505 9.295542e-01
16       - x8 0.2294200 9.298908e-01
17      - x16 0.2299119 9.591960e-01
18      - x24 0.2304502 9.795701e-01
19      - x23 0.2304744 9.806469e-01
20      - x25 0.2305981 9.846541e-01
21      - x18 0.2309955 9.905178e-01
22      - x14 0.2311176 9.935712e-01
23       - x7 0.2314548 9.958976e-01
24      - x15 0.2322827 9.991981e-01
25      - x11 0.2324040 9.993293e-01
26      - x12 0.2327577 9.996462e-01
\end{verbatim}
\end{footnotesize}

\doublespacing
In the truth, coeffs for \texttt{x1, x2, x3, x4, x5} are 1, others are 0.
\end{itemize}

\section{Estimation}

\subsection{Full model}
Fixed effect $R^2 = 0.613$, random effect $R^2 = 0.657$.

\paragraph{Summary}
\begin{scriptsize}
\singlespacing
\begin{verbatim}
Linear mixed model fit by REML ['lmerMod']
Formula: log(PRCP + 1) ~ LATITUDE + LONGITUDE + ELEVATION + TMAX + TMIN +  
    del_TT_Deg_Celsius + DMI + Nino34 + u_wind_200 + u_wind_600 +  
    u_wind_850 + v_wind_200 + v_wind_600 + v_wind_850 + X20E +  
    X70E + X80E + X100E + X120E + X140E + X160E + X120W + X40W +  
    X10W + NAO + EA + WP + EPNP + PNA + EAWR + SCA + TNH + POL +  
    SolarFlux + TempAnomaly + (1 | year)
   Data: rainsmall

REML criterion at convergence: 3669.2

Scaled residuals: 
    Min      1Q  Median      3Q     Max 
-3.8546 -0.6103  0.0413  0.6686  5.3352 

Random effects:
 Groups   Name        Variance Std.Dev.
 year     (Intercept) 0.1255   0.3543  
 Residual             0.9866   0.9933  
Number of obs: 1254, groups:  year, 35

Fixed effects:
                   Estimate Std. Error t value
(Intercept)         2.25454    0.06613   34.09
LATITUDE           -0.20304    0.13208   -1.54
LONGITUDE           0.22292    0.06087    3.66
ELEVATION          -1.05131    0.07017  -14.98
TMAX               -1.36674    0.05006  -27.30
TMIN                0.04238    0.08863    0.48
del_TT_Deg_Celsius  0.51886    0.09999    5.19
DMI                 0.10713    0.09945    1.08
Nino34             -0.21959    0.13555   -1.62
u_wind_200          0.12262    0.07379    1.66
u_wind_600          0.16603    0.07233    2.30
u_wind_850         -0.26071    0.09036   -2.89
v_wind_200         -0.03265    0.04539   -0.72
v_wind_600          0.03697    0.05347    0.69
v_wind_850         -0.29750    0.05613   -5.30
X20E               -0.41108    0.50330   -0.82
X70E               -0.28442    0.54449   -0.52
X80E               -0.29888    1.02230   -0.29
X100E              -0.21397    0.50682   -0.42
X120E              -0.12062    0.41540   -0.29
X140E              -0.27430    0.39681   -0.69
X160E              -0.38897    0.51338   -0.76
X120W              -0.89048    0.84230   -1.06
X40W                0.14006    0.44374    0.32
X10W               -0.11153    0.46846   -0.24
NAO                -0.08291    0.10330   -0.80
EA                 -0.13854    0.12878   -1.08
WP                  0.07865    0.10909    0.72
EPNP               -0.18460    0.14379   -1.28
PNA                -0.04457    0.10634   -0.42
EAWR                0.04610    0.09443    0.49
SCA                -0.02448    0.12695   -0.19
TNH                 0.15647    0.13868    1.13
POL                 0.18843    0.12327    1.53
SolarFlux          -0.14216    0.09213   -1.54
TempAnomaly         0.51812    0.13323    3.89
\end{verbatim}
\end{scriptsize}

\subsection{Reduced model}
Fixed effect $R^2 = 0.606$, random effect $R^2 = 0.649$.

\paragraph{Summary}
\begin{scriptsize}
\singlespacing
\begin{verbatim}
Linear mixed model fit by REML ['lmerMod']
Formula: log(PRCP + 1) ~ LATITUDE + LONGITUDE + ELEVATION + TMAX + del_TT_Deg_Celsius +  
    DMI + Nino34 + u_wind_200 + u_wind_600 + u_wind_850 + v_wind_850 +  
    X20E + X120W + NAO + EA + EPNP + TNH + POL + SolarFlux +      TempAnomaly + (1 | year)
   Data: rainsmall

REML criterion at convergence: 3656.4

Scaled residuals: 
    Min      1Q  Median      3Q     Max 
-3.7498 -0.6258  0.0296  0.6806  5.3763 

Random effects:
 Groups   Name        Variance Std.Dev.
 year     (Intercept) 0.1209   0.3478  
 Residual             0.9847   0.9923  
Number of obs: 1254, groups:  year, 35

Fixed effects:
                   Estimate Std. Error t value
(Intercept)         2.25535    0.06512   34.63
LATITUDE           -0.15626    0.12676   -1.23
LONGITUDE           0.25969    0.05225    4.97
ELEVATION          -1.08153    0.03886  -27.83
TMAX               -1.36575    0.04072  -33.54
del_TT_Deg_Celsius  0.48347    0.09609    5.03
DMI                 0.09566    0.06879    1.39
Nino34             -0.28544    0.10602   -2.69
u_wind_200          0.09826    0.07097    1.38
u_wind_600          0.16279    0.07041    2.31
u_wind_850         -0.24310    0.08651   -2.81
v_wind_850         -0.27575    0.04669   -5.91
X20E               -0.01084    0.10114   -0.11
X120W              -0.18115    0.12545   -1.44
NAO                -0.11622    0.07539   -1.54
EA                 -0.15751    0.07692   -2.05
EPNP               -0.22894    0.09024   -2.54
TNH                 0.23562    0.08035    2.93
POL                 0.12736    0.09691    1.31
SolarFlux          -0.09754    0.07652   -1.27
TempAnomaly         0.49256    0.08593    5.73
\end{verbatim}
\end{scriptsize}

\subsection{Full model vs. reduced model ANOVA}
\begin{footnotesize}
\singlespacing
\begin{verbatim}
          Df    AIC    BIC  logLik deviance  Chisq Chi Df Pr(>Chisq)  
mod.final 23 3620.9 3739.0 -1787.4   3574.9                           
mod.full  38 3627.0 3822.1 -1775.5   3551.0 23.847     15    0.06774 .
---
Signif. codes:  0 ‘***’ 0.001 ‘**’ 0.01 ‘*’ 0.05 ‘.’ 0.1 ‘ ’ 1
\end{verbatim}
\end{footnotesize}

\doublespacing
\section{Prediction}
We now use both the model with all variables and reduced set of variables to evaluate out-of-time prediction performance of resulting models. We use a rolling prediction scheme, in which for each of the years in 2003-2012, we use the previous 25 year's data to build the training model and test it to obtain predictions for yearly median rainfall for all 36 stations at the testing year.

In general predictions from full models are erratic, both in bias and MSE comparisons with the true values. Predictions from reduced models are more stable consistently. There seems to be slight positive bias from the reduced models, which are due to zero median precipitation values, which are always predicted positive.

\begin{figure*}
\captionsetup{justification=centering, font=footnotesize}
\begin{center}
\subfigure[]{\epsfxsize=0.45\linewidth \epsfbox{rolling_predbias_full_vs_reduced.png}}
\subfigure[]{\epsfxsize=0.45\linewidth \epsfbox{rolling_predMSE_full_vs_reduced.png}}\\
\subfigure[]{\epsfxsize=0.45\linewidth \epsfbox{rolling_density2012_full_vs_reduced.png}}
\subfigure[]{\epsfxsize=0.45\linewidth \epsfbox{rolling_map2012_full_vs_reduced}}
\caption{Comparing full model rolling predictions with reduced models: (a) Bias across years, (b) MSE across years, (c) density plots for 2012, (d) stationwise residuals for 2012}
\end{center}
\label{fig:prepost}
\end{figure*}

\begin{figure*}
\captionsetup{justification=centering, font=footnotesize}
\begin{center}
\subfigure[]{\epsfxsize=0.45\linewidth \epsfbox{rolling_predbias_full_vs_pvalreduced.png}}
\subfigure[]{\epsfxsize=0.45\linewidth \epsfbox{rolling_predMSE_full_vs_pvalreduced.png}}\\
\subfigure[]{\epsfxsize=0.45\linewidth \epsfbox{rolling_density2012_full_vs_pvalreduced.png}}
\subfigure[]{\epsfxsize=0.45\linewidth \epsfbox{rolling_map2012_full_vs_pvalreduced}}
\caption{Comparing full model rolling predictions with p-value reduced models: (a) Bias across years, (b) MSE across years, (c) density plots for 2012, (d) stationwise residuals for 2012}
\end{center}
\label{fig:prepostpval}
\end{figure*}

\end{document}