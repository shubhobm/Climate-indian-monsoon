\documentclass[11pt,twocolumn,twoside]{IEEEtran}
\usepackage{amsmath}
\usepackage[pdftex]{epsfig}
\usepackage{amsfonts}
\usepackage{amssymb}
\usepackage{fancyhdr}
\include{graphicsx}

\pagestyle{fancy}
%\renewcommand{\headrulewidth}{0pt}
\renewcommand{\footrulewidth}{0pt}
\rhead{\includegraphics[height=0.6in]{CITransparentwithTitleForWeb.png}}
\fancyhead[LO]{\sc Creating Post-Event Storm Tracks \ldots} % shorter form of title to fit in space
\fancyhead[LE]{\sc Lakshmanan, Herzog, Kingfield} % author list or et al., to fit in space
\chead{}
\cfoot{}

\begin{document}
\title{\vspace{0.2in}\sc Creating Post-Event Storm Tracks for Severe Weather Climatologies}
\author{Valliappa Lakshmanan$^{1,2}$\thanks{Corresponding author: V Lakshmanan, University of Oklahoma, Norman OK lakshman@ou.edu $^1$ Cooperative Institute of Mesoscale Meteorological Studies, University of Oklahoma $^2$ National Severe Storms Laboratory, Norman, OK}, Benjamin Herzog$^{1,2}$, Darrel Kingfield$^{1,2}$}

\maketitle
\thispagestyle{fancy}

\begin{abstract}
Commonly employed storm tracking algorithms do not use information on the
subsequent positions of a storm because it is not available at
the time that associations between frames are carried out,
but post-event analysis is
not similarly constrained. Therefore, it should be possible to
obtain better tracks for post-event analysis than what a real-time
algorithm is capable of. In this paper, we describe a statistical
procedure to determine storm tracks from a set of identified storm cells
over time. We find that this procedure results in fewer, longer-lived tracks
at all scales.
\end{abstract}

\section{Motivation}
Even though storm tracking methods such as the Storm Cell Identification
and Tracking Algorithm (SCIT~\cite{scit}), Thunderstorm Identification,
Tracking and Nowcasting (TITAN~\cite{titan}) and Segmentation-Motion
Estimation (w2segmotion \cite{atmosresearch,stormattr}) are constrained
to work in a purely causal fashion, these algorithms have been widely
employed by the meteorological research community to carry out case
studies and formulate spatio-temporal relationships, for example by~\cite{scituse1,scituse3,segmotionuse1,titanuse1}.

Using a storm tracking algorithm that is constrained to work in real-time
to carry out post-event analysis is sub-optimal. There is more information
(about which cells persist and the direction in which they move)
that is available if the entire set of storm
cell identifications over the complete dataset is used to determine
thunderstorm tracks. In this paper, we describe a way of clustering
a set of storm cell identifications over time into trajectories where
a trajectory is the line (or curve) that best fits the
position of an individual storm cell over time.

This work was carried out in order to improve spatiotemporal relationships
between radar-derived storm characteristics and the subsequent onset of
specific weather hazards such as cloud-to-ground lightning, hail and
tornadoes~\cite{phi}.
Such hazard probabilities can be derived from storm attributes
using the method of~\cite{stormattr} on a multi-year reanalysis dataset created
as described in~\cite{hailclimo}, but the reliability and skill of
these probabilities is limited by the quality of the storm tracks used
to train the data mining algorithms.

\section{Method}
Given a cluster of storm cells $(x_t,y_t)$ at multiple times,
the best constant-speed straight-line trajectory fit $u,v$
for the cluster is the best fit slope of the line that connects the
points in the cluster.
\cite{theil} introduced a non-parametric, rank-invariant method for
obtaining the best-fit slope in a dataset whereby one computes
the median of the slopes of every pair of sample points.
\cite{sen} modified the definition so that the median is computed
only of points at different times ($t_2 \ne t_1$).
Once the median of the slopes ($u$ and $v$) are obtained, and
assuming that $t_0$ is the time of the earliest storm cell in the cluster,
the value of $x_0$ can be obtained by computing the median value of
$x(t) - u(t-t_0)$ over all the storm cells in the cluster.
This value was shown by~\cite{sen} to be the value that makes the
Kendall rank correlation coefficient~\cite{kendalltau} between
the actual storm cell locations and the fitted values on the line
approximately zero.

The clustering method we use is a variant of K-Means clustering where
the cluster center is defined to be Thiel-Sen fit to the set of points
in the cluster and distance between a storm cell at $(x,y,t)$
and the cluster is
defined to be the Euclidean distance between the storm cell location and the
Theil-Sen estimate at that time.

\begin{figure}
\begin{center}
\epsfxsize=0.95\hsize \epsfbox{trackstats.png} 
\end{center}
\caption{Effect of clustering storm tracks at different scales.}
\label{trackstats.fig}
\end{figure}

The clustering method is as follows:
\begin{enumerate}
\item Find an initial estimate of tracks in the dataset. This can
be obtained from any robust storm tracking algorithm, even a real-time
one such as that of~\cite{scit,titan,stormattr}.
\item Treating each track (set of storm cells with the same id) as
a cluster, compute the Theil-Sen slope and constants ($u,v,x_0,y_0,t_0$)
for each cluster.
\item For every storm cell in the dataset, find the nearest cluster.
If the nearest cluster is different from the cluster the cell is
currently part of, and if the distance is less than some reasonable
threshold $D$, move the storm cell to the nearest cluster.
\item Compute the Theil-Sen fit for each cluster, prune the set of
clusters to remove substantially identical trajectories and carry out Step 3,
repeating steps 3 and 4
until there are no more changes or until the number of iterations
reaches some maximum (we used 3 iterations as this maximum number).
\end{enumerate}

\section{Evaluation}
Following~\cite{scoretrack}, we carried out a statistical analysis of
the set of storm tracks extracted from the radar data of June 17, 2012.
At the most detailed (200 $km^2$)
scale, the number of trajectories is cut by about
a third as a result of postanalysis (See Figure~\ref{trackstats.fig}).
The error in size fit (computed by
fitting the sizes of the storm cells within a trajectory to a
``growth-and-decay'' parabola
and looking for deviations from that fit -- see~\cite{scoretrack} for
details), an indicator
of how likely it is that two separate tracks are wrongly combined,
increases by a very small amount. The position error, an indicator
of how likely it is that storm cells are added to tracks they are not
part of, also increases but remains limited to be below the 0.1 decimal
degree limit imposed by $D$. The fourth panel of Figure~\ref{trackstats.fig}
demonstrates the benefit of postanalysis -- the mean duration of the
tracks increases by about 50\%, from an average of about 2000 seconds
to an average of over 3000 seconds. At the moderate (600 $km^2$ scale)
and coarse (1000 $km^2$ scale), the behavior is similar.
For a very small cost in terms of potentially wrong associations, one gets
a significant improvement in the form of longer-lived tracks.

\section*{Acknowledgments}
Funding for the authors was provided by NOAA/Office of Oceanic and Atmospheric Research under NOAA-OU Cooperative Agreement NA11OAR4320072, U.S. Department of Commerce.

\bibliographystyle{ieeetr}
\bibliography{bibfile,benherzog}

\end{document}
